\section{Introducció i Objectius}
El present treball forma part de l'assignatura de Sistemes de Propulsió d'Aeronaus. Gran part d'aquesta assignatura consisteix en l'estudi dels tipus de motors d'una aeronau i de les possibilitats d'optimització, a més de la parametrització dels motors tant en cas ideal com en cas real.\\
Per tal profunditzar en les àrees de coneixement relacionades amb l'assignatura es proposa la realització d'aquest projecte. L'objectiu es el disseny preliminar de la motorització d'una aeronau partint de tres condicions de disseny: empenta, velocitat i altitud de vol. Amb aquestes condicions el sistema no queda definit, de manera que es necessari establir certs criteris per a aconseguir tots els paràmetres del motor. En les següents pàgines es discutirà quin tipus de motor pot ser adequat i el criteri de disseny a utilitzar.\\
A més de fer el càlcul del motor escollit, es faran estudis per a la possible implementació de \textit{mixer}, \textit{propeller} i \textit{afterburner}. Amb aquests estudis es busca saber si pel motor escollit aquests sistemes son profitosos o no, tenint en compte els resultats obtinguts amb ells i sense ells i el pensament crític del grup de treball.\\
Per últim es discutirà com serà el motor final i es calcularan les àrees d'aquest i el consum d'aire i combustible.\\
Els càlculs que es mostraran han estat realitzats utilitzant un codi desenvolupat en MATLAB. Aquest codi ha sigut verificat utilitzant com a referencia problemes resolts a classe\footnote{Verificació del codi es pot visualitzar a l'Annex B} per a assegurar la validesa dels resultats. El codi es pot visualitzar a l'Annex A. La informació teòrica per a poder-lo fer a estat extreta dels apunts de classe de teoria i problemes de l'assignatura de Sistemes de Propulsió d'Aeronaus tant per part de Pau Manent com per part de Marc Maymó i del llibre \textit{Elements of Gas Turbine Propulsion}, de Jack D. Mattingly. 
\section{Càlcul i elecció de postcombustor}
Com s'ha vist amb anterioritat, el mixer pe'ls nostres paràmetres donava problemes a l'implementar-lo. Així doncs, es decideix calcular un postcombustor que només vagi adherit al core del turbofan, creant la postcombustió del flux primari.

\subsection{Paràmetres suposats }
Tot i que la gran majoria de paràmetres venien donats per l'enunciat del treball, amb l'objectiu de fer-lo més realista s'han suposat certes eficiències específiques del component estudiat.

TAULA PARAMETRES

\subsection{Posctombustor adherit al flux primari}
Per calcular el postcombustor, s'ha seguit la referencia dels problemes solucionats per Pau Manent a l'assignatura i la referencia del llibre \cite{mattingly}.

\noindent Es comença solucionant el cas del turbofan real, tal i com s'a implementat al codi, fins obtenir els valors dels seus paràmetres a les toveres. Després, s'aplica el càlcul del postcombustor, situat al final de la turbina. Finalment, les seccions situades després del postcombustor són recalculades ja que aquest actua sobre elles.

PROCÉS ESPECÍFIC Càlcul

\subsubsection{Resultats postcombustor}


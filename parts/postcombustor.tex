\clearpage
\section{Càlcul i elecció de postcombustor}
Com s'ha vist amb anterioritat, el mixer pe'ls nostres paràmetres donava problemes a l'implementar-lo. Així doncs, es decideix calcular un postcombustor que només vagi adherit al core del turbofan, creant la postcombustió del flux primari.

\subsection{Paràmetres suposats }
Tot i que la gran majoria de paràmetres venien donats per l'enunciat del treball, amb l'objectiu de fer-lo més realista s'han suposat certes eficiències específiques del component estudiat.

\begin{figure}[H]
	\centering
	\begin{tabular}{lc}
		\toprule[3pt]
		\textbf{Paràmetre}&\textbf{Valor}\\
		\midrule[1pt]
		$\gamma_{AB}$ & $1.3$\\
		$ Cp_{AB}$ & $Cp_t$\\
		$T_{t7}$ & $2400K$\\
		$\eta_{AB}$ & $0.99$\\
		
		\bottomrule[2pt]
	\end{tabular}
	\label{ABparam}
	\caption{Paràmetres suposats al postcombustor.}
\end{figure}


\subsection{Postcombustor adherit al flux primari}
Per calcular el postcombustor, s'ha seguit la referencia dels problemes solucionats per Pau Manent a l'assignatura i la referencia del llibre \cite{mattingly}.\\

\noindent Es comença, solucionant el cas del turbofan real com en els apartats anteriors, fins obtenir els valors dels seus paràmetres a les toveres.

\noindent Després, comença el càlcul del postcombustor, situat al final de la turbina. Es considera que $T_{t9} = T_{t7}$, $P_{t9}=\pi_nP_{t6}$ i es defineix $\tau_{\lambda AB} = \frac{Cp_{AB}T_{t7}}{Cp_cT_0}$.

\noindent Aleshores, es calcula la fracció de massa de combustible ($f_{AB}=\dot{m}_{fAB}/\dot{m}_0$) que el postcombustor afegeix al flux primari.

\begin{equation}
	f_{AB}=(1+f)\frac{\tau_{\lambda AB}-\tau_{\lambda}\tau_t}{\eta_{AB}\frac{h}{Cp_cT_0}-\tau_{\lambda AB}}
\end{equation}
 
 
\noindent Finalment, les seccions situades després del postcombustor són recalculades, ja que, els paràmetres de sortida del postcombustor les modifica. Concretament s'imposa que $f = f +f_{AB}$, dins el codi per poder aplicar la funció de càlcul de la força adimensional (\textit{Fadimensional.m}). S'acaba el càlcul del postcombustor extraient els cabals màssics característics del motor amb postcombustor (\textit{Fluxosmasics.m}).

\subsubsection{Resultats postcombustor}
A la següent taula, es compara el motor dissenyat amb i sense postcombustor, per decidir sobre la seva eficàcia.
\begin{figure}[H]
	\centering
	\begin{tabular}{lcc}
		\toprule[3pt]
		\textbf{Paràmetre}&\textbf{Valor amb AB}&\textbf{Valor sense AB}\\
		\midrule[1pt]
		$\hat{F}$ & $7.02$ & $5.50$\\
		$ \dot{m}_0$ & $11.80kg/s$ & $15.06kg/s$\\
		$ \dot{m}_f$ & $0.90kg/s$ & $0.56kg/s$\\
		$ f$ & $0.0775$ & $0.0369$\\
		$D_0$ & $82.99cm$ & $93.76cm$\\
		$D_9$ & $35.63cm$ & $34.01cm$\\
		$T_{t9}$ & $2400K$ & $1320K$\\
		
		\bottomrule[2pt]
	\end{tabular}
	\label{ABres}
	\caption{Resultats d'implementar el postcombustor.}
\end{figure}

\noindent La comparativa de resultats, dona informació molt interessant. Per començar, amb el postcombustor la força adimensional ($\hat{F}$)  augmenta. Podem explicar aquest comportament, perquè $ \dot{m}_0$ disminueix i $\hat{F}=F/(\dot{m}_0a_0)$. Qualitativament, el fet d'afegir massa de combustible (més dens que l'aire) i treure l'aire per la tovera a una temperatura més elevada, provoca que amb menys massa ejectada (però amb més densitat), s'aconsegueixi la mateixa empenta.

\noindent Tanmateix, el cabal màssic de combustible total augmenta, perquè tot i que $ \dot{m}_f$ és aproximadament un terç més baix, el paràmetre $ f$ és més del doble.

\noindent Per últim, la variació d'àrees de sortida i entrada no es significativa, tot i que podem destacar, que l'àrea d'entrada es menor degut a que necessitem un flux d'aire menor, però la de sortida es major, ja que ejectem partícules més denses a més temperatura que sense el postcombustor.

\noindent Com a conclusió d'aquest apartat, podríem afirmar que posar un postcombustor surt a compte sempre i quant, el consum del combustible no suposi un problema; ja que, amb postcombustor aquest motor consumeix prop del doble de combustible que sense i, a canvi, el motor només redueix la seva envergadura a l'entrada en 10cm. Si no hi ha uns condicionants geomètrics molt restrictius, no consideraríem utilitzar un postcombustor al turbofan. 


\clearpage


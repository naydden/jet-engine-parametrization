\section{Càlcul i elecció de postcombustor}
Com s'ha vist amb anterioritat, per als nostres paràmetres de disseny el \textit{mixer} no donava valors profitosos i s'ha decidit no implementar-lo. Així doncs, es decideix calcular un postcombustor que només vagi adherit al \textit{core} del \textit{turbofan}, creant la postcombustió del flux primari.

\subsection{Paràmetres suposats }
Tot i que la gran majoria de paràmetres venien donats per l'enunciat del treball, amb l'objectiu de fer-lo més realista s'han suposat certes eficiències específiques i una altra temperatura de sortida del component a estudiar.

\begin{figure}[H]
	\centering
	\begin{tabular}{lc}
		\toprule[3pt]
		\textbf{Paràmetre}&\textbf{Valor}\\
		\midrule[1pt]
		$\gamma_{AB}$ & $1.3$\\
		$ Cp_{AB}$ & $Cp_{hot}$\\
		$T_{t7}$ & $2400K$\\
		$\eta_{AB}$ & $0.99$\\
		
		\bottomrule[2pt]
	\end{tabular}
	\label{ABparam}
	\caption{Paràmetres suposats al postcombustor.}
\end{figure}


\noindent Per calcular el postcombustor es comença solucionant el cas del \textit{turbofan} real com en els apartats anteriors, fins obtenir els valors dels seus paràmetres a la sortida de la turbina de baixa pressió.

\noindent Després, comença el càlcul del postcombustor. Es considera que $T_{t9} = T_{t7}$, $P_{t9}=\pi_nP_{t6}$ i es defineix $\tau_{\lambda AB} = \frac{Cp_{AB}T_{t7}}{Cp_cT_0}$.

\noindent Aleshores, es calcula la fracció de massa de combustible ($f_{AB}=\dot{m}_{fAB}/\dot{m}_0$) que el postcombustor afegeix al flux primari. De igual manera que en el cas de la cambra de combustió ja existent, no s'ha considerad que $(1+f)=1$.

\begin{equation}
	f_{AB}=(1+f)\frac{\tau_{\lambda AB}-\tau_{\lambda}\tau_t}{\eta_{AB}\frac{h}{Cp_cT_0}-\tau_{\lambda AB}}
\end{equation}
 
 
\noindent Les seccions situades després del postcombustor són recalculades ja que els paràmetres de sortida del postcombustor les modifica. Es calculen doncs les toveres del primari i del secundari i es re-defineix $f = f +f_{AB}$ dins el codi per poder aplicar la funció de càlcul de la força adimensional (\textit{Fadimensional.m}). S'acaba el càlcul del postcombustor extraient els cabals màssics característics del motor amb postcombustor (\textit{Fluxosmasics.m}).

A la següent taula, es compara el motor dissenyat amb i sense postcombustor, per decidir sobre la seva eficàcia.
\begin{table}[H]
	\centering
	\resizebox{8cm}{!}{
		\begin{tabular}{lcc}
		\toprule[3pt]
		\textbf{Paràmetre}&\textbf{Valor amb AB}&\textbf{Valor sense AB}\\
		\midrule[1pt]
		$\hat{F}$ & $8.6803$ & $6.2908$\\
		$ \dot{m}_0$ & $9.55kg/s$ & $13.18kg/s$\\
		$ \dot{m}_f$ & $0.81kg/s$ & $0.48kg/s$\\
		$ f$ & $0.0850$ & $0.0365$\\
		$D_0$ & $93.06cm$ & $109.32cm$\\
		$D_9$ & $61.56cm$ & $58.12cm$\\
		$T_{t9}$ & $2400K$ & $1017K$\\
		
		\bottomrule[2pt]
	\end{tabular}}
	\label{ABres}
	\caption{Resultats d'implementar el postcombustor.}
\end{table}


\noindent La comparativa de resultats, dona informació molt interessant. Per començar, amb el postcombustor la força adimensional ($\hat{F}$)  augmenta. Podem explicar aquest comportament, perquè $ \dot{m}_0$ disminueix i $\hat{F}=F/(\dot{m}_0a_0)$. Qualitativament, el fet d'afegir massa de combustible (més dens que l'aire) i treure l'aire per la tovera a una temperatura més elevada, provoca que amb menys massa ejectada (però amb més densitat), s'aconsegueixi la mateixa empenta.

\noindent Tanmateix, el cabal màssic de combustible total augmenta, perquè tot i que $ \dot{m}_0$ amb postcombustor és aproximadament un terç més baix  que $ \dot{m}_0$ sense, el paràmetre $ f$ és més del doble pel cas del postcombustor.

\noindent Per últim, la variació d'àrees de sortida i entrada no es significativa, tot i que podem destacar que l'àrea d'entrada amb postcombustor es menor degut a que necessitem un flux d'aire $ \dot{m}_0$ menor.

\noindent Ara bé, l'àrea de sortida es major al posar el postcombustor, ja que, ejectem partícules més denses i a més temperatura que sense el postcombustor.

\noindent Com a conclusió d'aquest apartat, podríem afirmar que posar un postcombustor surt a compte sempre i quan el consum del combustible i la complexitat del motor no suposi un problema (pocs cops passa); ja que, amb postcombustor aquest motor consumeix prop del doble de combustible que sense i, a canvi, el motor només redueix la seva envergadura a l'entrada en 16cm. Si no hi ha uns condicionants geomètrics molt restrictius, no es consideraria utilitzar un postcombustor al \textit{turbofan}. 


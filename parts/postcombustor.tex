\section{Càlcul i elecció de postcombustor}
Com s'ha vist amb anterioritat, el mixer pe'ls nostres paràmetres donava problemes a l'implementar-lo. Així doncs, es decideix calcular un postcombustor que només vagi adherit al core del turbofan, creant la postcombustió del flux primari.

\subsection{Paràmetres suposats }
Tot i que la gran majoria de paràmetres venien donats per l'enunciat del treball, amb l'objectiu de fer-lo més realista s'han suposat certes eficiències específiques del component estudiat.

\begin{figure}[H]
	\centering
	\begin{tabular}{lc}
		\toprule[3pt]
		\textbf{Paràmetre}&\textbf{Valor}\\
		\midrule[1pt]
		$\gamma_{AB}$ & $1.3$\\
		$ Cp_{AB}$ & $Cp_c$\\
		$T_{t7}$ & $2400K$\\
		$\eta_{AB}$ & $0.99$\\
		
		\bottomrule[2pt]
	\end{tabular}
	\label{ABparam}
	\caption{Paràmetres suposats al postcombustor.}
\end{figure}


\subsection{Postcombustor adherit al flux primari}
Per calcular el postcombustor, s'ha seguit la referencia dels problemes solucionats per Pau Manent a l'assignatura i la referencia del llibre \cite{mattingly}.\\

\noindent Es comença, solucionant el cas del turbofan real com en els apartats anteriors, fins obtenir els valors dels seus paràmetres a les toveres.

\noindent Després, comença el càlcul del postcombustor, situat al final de la turbina. Es considera que $T_{t9} = T_{t7}$, $P_{t9}=\pi_nP_{t6}$ i es definieix $\tau_{\lambda AB} = \frac{Cp_{AB}T_{t7}}{Cp_cT_0}$.

\noindent Aleshores, es calcula la fracció de massa de combustible ($f_{AB}=\dot{m}_{fAB}/\dot{m}_0$) que el postcombustor afegeix al flux primari.

\begin{equation}
	f_{AB}=(1+f)\frac{\tau_{\lambda AB}-\tau_{\lambda}\tau_t}{\eta_{AB}\frac{h}{Cp_cT_0}-\tau_{\lambda AB}}
\end{equation}
 
 
\noindent Finalment, les seccions situades després del postcombustor són recalculades ja que els paràmetres d'qeuest les modifica.



PROCÉS ESPECÍFIC Càlcul

\subsubsection{Resultats postcombustor}


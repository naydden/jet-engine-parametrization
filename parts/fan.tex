\section{Estudi implementació \textit{Propeller}}
Per escollir una hèlix, cal calcular el \textit{ turboprop} que la mourà. Al \textit{turbofan} dissenyat anteriorment, a nivell pràctic, se l'hi extraurà el fan per substituir-lo per una hèlix. En aquesta secció s'explicaran els dos mètodes implementats per calcular el \textit{turboprop}: el primer consisteix en optimitzar la turbina de baixa pressió i el segon en simplement canviar el fan per una hèlix sense optimització, mantenint constant la turbina implementada.

\noindent A nivell d'implementació, s'han incorporat subrutines a dins de les funcions de càlcul del \textit{turbofan} per tenir en compte els elements addicionals estudiats. 
\subsection{Càlcul del \textit{turboprop} amb optimització de la potència de la turbina de baixa pressió}
Es comença calculant el \textit{turbojet} que alimenta l'hèlix, seguint els mateixos passos que en el turbofan des de el difusor a la tovera de sortida.

\noindent Les expressions del difusor no es veuen alterades respecte les del \textit{turbofan}. En canvi, per calcular el compressor cal tenir en compte que no hi ha \textit{fan}, per la qual cosa $P_{t25}=P_{t2}$, $T_{t25}=T_{t2}$ i $\pi_{cH}=\pi_{C}$. S'ha considerat que la $\pi_c$ total del compressor del \textit{turbofan}, ara ho sigui del compressor del \textit{turboprop}.

\noindent Es segueix parametritzant el motor per la cambra de combustió, on es desenvolupen les mateixes expressions que pel turbofan amb l'excepció que $\tau_f=1$.

\noindent La turbina d'alta pressió aportarà el treball per moure el compressor i es calcula com en el cas del \textit{turbofan}. D'altre banda, la turbina de baixa pressió ja no està lligada al fan (s'ha tret), sinó que alimentarà l'hèlix. La idea del \textit{turboprop} es que la major part de l'empenta s'aconsegueixi mitjançant el moviment de més massa d'una manera més lenta per part de l'hèlix. Per aquesta raó el que interessa es obtenir empenta per part de l'hèlix i no per part del \textit{turbojet}, de manera que es busca expansionar el fluid a la turbina de baixa el màxim possible maximitzant la potència transmesa a l'hèlix. 
\noindent Així doncs, alguns paràmetres de la turbina de baixa es calcularien segons:
\begin{align}
	\tau_{tL}&=\frac{1}{\tau_{cH}\tau_r}\frac{(\tau_r\tau_{cH}(\tau_r-1)+\tau_\lambda)}{\tau_\lambda-\tau_r\tau_{c}+\tau_r}\label{turboprop_turbopt}\\
	T_{t5}&=\tau_{tL}T_{t45}
\end{align}

\noindent El valor de $\tau$ serà el valor real mentre que el rati de pressió resultant ha de ser calculat tenint en compte l'eficiència de la turbina, tal i com s'ha mostrat a l'apartat \ref{motoreal}. \\
\\
Per calcular la tovera, s'assumeix que $P_{t6}=P_{t5}$ i $T_{t6}=T_{t5}$. Per als paràmetres $\pi_c$, $\pi_f$ i $\alpha$, escollits en l'optimització resulta que el Mach de sortida, $M_9$ es imaginari. Aquest resultat es degut a que l'equació \ref{turboprop_turbopt} obtinguda de la referencia \cite{mattingly} no té en compte les eficiències de la resta de components del motor per a calcular el punt de màxima expansió de fluid a la turbina.  
\noindent Per solucionar aquest problema, s'implementa una subrutina a la funció \textit{ToveraPrimari.m} amb la següent idea: idealment s'expansionaria tot el fluid a la turbina de baixa pressió per donar el màxim de potencia a l'hèlix però deixant el fluid amb l'energia suficient per a sortir del \textit{turbojet}. Es proposa fixar un Mach de sortida molt baix, $M_9 = 0.1$ i seguidament fer una  re-calculació de $\tau_{tL}$ per aquest Mach. El resultat, es una nova $\tau_{tL}$ que no és la més optima per a condicions ideals però es propera a la optima per a condicions reals.

\noindent Finalment, es calculen els paràmetres característics del \textit{turboprop}:
\begin{multline}
	C_{cin} = (\gamma_c-1)M_0\cdot\\
	\left[\sqrt{\frac{2}{\gamma_c-1}\left(\tau_{tL}(\tau_{\lambda}+\tau_r-\tau_r\tau_{cH})-\frac{\tau_{\lambda}}{\tau_r\tau_{cH}}\right)}-M_0\right] \label{turboprop_eqn1}
\end{multline}
	\begin{align}
	C_{prop} &= \eta_{prop}\eta_{mec}(1-\tau_{tL})(\tau_\lambda+\tau_r-\tau_r\tau_{cH}) \label{turboprop_eqn2}\\
	C_{tot} &= C_{cin} + C_{prop} \label{turboprop_eqn3}
	\end{align}
S'obtenen els següents resultats:
\begin{table}[H]
	\centering
	\begin{tabular}{lc}
		\toprule[3pt]
		\textbf{Paràmetre}&\textbf{Valor}\\
		\midrule[1pt]
		$\tau_{tL}$ & 0.8168\\
		$C_{cin}$ & 0.6113\\
		$C_{prop}$ & 0.9801\\
		$C_{tot}$ & 1.5914\\
		
		\bottomrule[2pt]
	\end{tabular}
\label{C_opti}
\caption{Paràmetres C turboprop amb turbina de baixa pressió optimitzada}
\end{table}
\noindent Tot i que $C_{cin}=0.6113$, està lluny del cas ideal on seria 0; amb $C_{prop}>C_{cin}$, s'entén que la font majoritària d'empenta és l'hèlix, representant un 62\% de l'empenta del \textit{turboprop}.

\subsection{Càlcul del turboprop substituint el fan per una hèlix}
En aquesta secció es proposa un altra idea per a la implementació de l'hèlix. Un cop calculats els paràmetres de la turbina del \textit{turbofan}, es guarden $\tau_{tL}$ i $\pi_{tL}$  per utilitzar-los dins la subrutina del \textit{turboprop}, ja que es considera que l'únic que es fa es treure el \textit{fan} i posar una hèlix sense variar la turbina que proporciona el treball necessari. Aleshores, es calculen tots el paràmetres del \textit{turboprop} com en l'apartat anterior, amb la diferència de que ara, no s'optimitza la turbina de baixa pressió, es deixa la mateixa que s'ha calculat pel \textit{turbofan}.

\noindent Els resultats en aquest cas son els següents:
\begin{table}[H]
	\centering
	\begin{tabular}{lc}
		\toprule[3pt]
		\textbf{Paràmetre}&\textbf{Valor}\\
		\midrule[1pt]
		$\tau_{tL}$ & 0.8751\\
		$C_{cin}$ & 0.6680\\
		$C_{prop}$ & 0.6681\\
		$C_{tot}$ & 1.3362\\
		
		\bottomrule[2pt]
	\end{tabular}
	\label{C_opti2}
	\caption{Paràmetres C turboprop amb turbina de baixa pressió no optimitzada}
\end{table}
\noindent $C_{prop}\approx C_{cin}$. Tal i com calia esperar, el fet de no optimitzar $\tau_{tL}$, fa que el fluid s'expandeixi menys ($\tau_{tL}$ major) a la turbina de baixa pressió, fent que hi hagi menys potència disponible per l'hèlix. Conseqüentment, la empenta mitjançant l'hèlix no representa més del 50\% del conjunt motor total.

\subsection{Conclusions sobre el \textit{turboprop}}
Mitjançant l'anàlisi d'aquests tipus de motor, s'ha vist com d'important és optimitzar la turbina de baixa pressió.

\noindent Ara bé, donat que en cap dels dos casos s'aconsegueix extreure la suficient potencia del fluid per utilitzar l'hèlix del \textit{turboprop} com a quasi-únic element de propulsió, no es recomana implementar el \textit{turboprop} en vers del \textit{turbofan} perquè amb les característiques calculades, es tractaria d'un motor poc eficient.

\subsection{Elecció de l'hèlix}
Tot i que s'ha decidit no implementar un \textit{turboprop} per les raons anteriorment esmentades, si que s'ha desenvolupat un algoritme per tal d'escollir l'hèlix a utilitzar. Per tal de fer-ho s'ha considerat el comportament que es pot visualitzar a la Figura \ref{diag_helix}.
\begin{figure}[H]
	\centering
	\includegraphics[width=0.5\textwidth]{./pics/parametreshelix.JPG}
	\caption{Corbes de coeficient de potència d'una hèlix de pas variable.}
	\label{diag_helix}
\end{figure}
\noindent Del càlcul del \textit{turboprop} obtenim l'empenta i la potència de l'hèlix, amb la qual cosa el que s'hauria de definir utilitzant la gràfica és el diàmetre i les revolucions a les quals gira l'hèlix. Per entrar a la gràfica es necessiten els següents coeficients: 
\begin{align}
	C_{P} &= \frac{P}{\rho n^3 D^5} \\
	C_{T} &= \frac{T}{\rho n^2 D^4} 
	\end{align}
Amb els quals s'obtindria el paràmetre d'avanç: 
\begin{align}
	J &= \frac{V}{nD}  
	\end{align}
Degut a que aquests coeficients també depenen de les revolucions i el diàmetre, es necessari elaborar un procés iteratiu per tal de cercar possibles hèlixs que siguin solució. Per fer-ho s'ha creat una taula de valors a MATLAB on es relacionin $C_P$, $C_T$ i $J$ d'acord amb la gràfica \ref{diag_helix}. Per tal de simplificar el procés s'ha escollit la línia de pas de 40$^{\circ}$ per fer la taula. Seguidament s'ha iterat en valors de $n$ i $D$ fins a trobar els coeficients de potència, empenta i avanç que es corresponguessin a un punt de la gràfica definida. El resultat ha sigut el següent: 
 \begin{table}[H]
	\centering
	\begin{tabular}{lc}
		\toprule[3pt]
		\textbf{Paràmetre}&\textbf{Valor}\\
		\midrule[1pt]
		$C_{p}$ & 0.195\\
		$C_{T}$ & 0.2491\\
		$J$ & 0.7349\\
		$n (rev/s)$ & 126\\
		$D (m)$ & 1.8\\
		\bottomrule[2pt]
	\end{tabular}
	\caption{Paràmetres de l'hèlix.}
\end{table}
\noindent Al fer la iteració ens adonem que es molt difícil trobar valors que corresponguin als de la gràfica. S'hauria de tenir més informació, com per exemple tots els possibles angles de pas implementats a la taula per a fer la iteració i trobar valors adequats. 
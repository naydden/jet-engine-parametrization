\section{Descripició del/s motor/s}
\label{descripcio}
Les característiques que ha de complir el motor son les següents:
\begin{itemize}
\item Empenta de creuer: $F=25000N$
\item Alçada de vol: $h=9500m$
\item Velocitat de creuer: $v=600km/h$
\end{itemize}
A més de tot això es tracta d'un disseny real, per la qual cosa s'han de tenir en compte certs ratis de pressió i temperatura que afectaran com a eficiències. Tant les dades d'eficiència com la temperatura màxima a l'entrada de la turbina poden ser consultades a l'enunciat del projecte.
%\begin{itemize}
%\item Rati de pressió al difusor: $\pi_d=0.96$
%\item Rendiment al compressor: $\eta_c=0.88$
%\item Rati de pressió a la cambra de combustió: $\pi_b=0.94$
%\item Eficiència de combustió: $\eta_b=0.99$
%\item Rendiments de turbina: $\eta_{tH}=\eta_{tL}=0.87$
%\item Rati de pressió a la tovera: $\pi_n=0.98$
%\item Rendiment mecànic: $\eta_{mec}=0.99$
%\item Temperatura d'entrada a la turbina: $T_{t4}=1780$
%\end{itemize}
Segons la Figura \ref{operacio}, per a les característiques esmentades els millors candidats serien el turbojet, el turbofan i el turboprop, tot i que aquest últim està bastant restringit a augmentar la velocitat de vol.
\begin{figure}[H]
\centering
\includegraphics[width=0.5\textwidth]{./pics/seleccio.JPG} 
\caption{Operació motors d'aeronaus. Imatge extreta de \cite{mattingly}.}
\label{operacio}
\end{figure}
\noindent Un dels criteris per al disseny del motor serà el de minimitzar el consum de combustible incrementant l'eficiència propulsiva. Això es podria aconseguir tant amb un turboprop com amb un turbofan, ja que aquests mouen més massa a una velocitat més petita. La velocitat de creuer que es requisit de disseny es prou baixa per a un turboprop, tot i així, es decideix dissenyar primerament un turbofan per a augmentar la capacitat de velocitat de l'aeronau i poder-la equiparar a aeronaus actuals, amb una velocitat de creuer generalment més alta. 
\section{Descripició del/s motor/s}
\label{descripcio}
Les característiques que ha de complir el motor son les següents:
\begin{itemize}
\item Empenta de creuer: $F=25000N$
\item Alçada de vol: $h=9500m$
\item Velocitat de creuer: $v=600km/h$
\end{itemize}
A més de tot això es tracta d'un disseny real, per la qual cosa s'han de tenir en compte els següents ratis i temperatura màxima d'entrada a la turbina:
\begin{itemize}
\item Rati de pressió al difusor: $\pi_d=0.96$
\item Rendiment al compressor: $\eta_c=0.88$
\item Rati de pressió a la cambra de combustió: $\pi_b=0.94$
\item Eficiència de combustió: $\eta_b=0.99$
\item Rendiments de turbina: $\eta_{tH}=\eta_{tL}=0.87$
\item Rati de pressió a la tovera: $\pi_n=0.98$
\item Rendiment mecànic: $\eta_{mec}=0.99$
\item Temperatura d'entrada a la turbina: $T_{t4}=1780$
\end{itemize}
Segons la Figura \ref{operacio}, per a les característiques esmentades els millors candidats serien el turbojet i el turbofan.
\begin{figure}[h]
\centering
\includegraphics[width=0.5\textwidth]{./pics/seleccio.JPG} 
\caption{Operació motors d'aeronaus. Imatge extreta de \cite{mattingly}.}
\label{operacio}
\end{figure}
Un dels criteris per al disseny del motor serà el de minimitzar el consum de combustible incrementant l'eficiència propulsiva. El tipus de motor adequat es doncs el turbofan,  més massa a una velocitat més petita. 